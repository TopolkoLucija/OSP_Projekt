\chapter{Zaključak}

U ovom projektu detaljno smo analizirala skup podataka o filmovima s portala IMDb, primjenjujući osnove statističkog programiranja i razvijajući prediktivne modele temeljene na strojnom učenju. Kroz proces pregleda, čišćenja podataka te statističke analize i vizualizacije, stekle smo duboko razumijevanje atributa skupa podataka, međusobnih odnosa te potencijalnih trendova.

Proučavale smo brojne aspekte, uključujući najznačajnije glumce, redatelje, godine premijere, budžet, zaradu i prosječne ocjene korisnika. Osim toga, istražile smo najčešće ključne riječi i žanrove, dobivajući uvid u preferencije publike. Također, razvile smo prediktivne modele za procjenu uspješnosti filma koristeći metode potpornih vektora i slučajne šume.

U analizi međusobne ovisnosti atributa primijetile smo trendove koji ukazuju na povezanost između broja glasova, recenzija te ocjena filmova. 

U razvoju prediktivnih modela, postigle smo zadovoljavajuće rezultate, s uspješnosti od 75.6\% preciznosti za metodu potpornih vektora i 90.9\% za metodu slučajne šume. Važno je napomenuti da je uključivanje dodatnih atributa ili isprobavanje drugih metoda optimizacije moglo poboljšati performanse modela.

Analiza skupa podataka o filmovima s IMDb portala pružila nam je vrijedan uvid u karakteristike filmova te nam omogućila razvoj prediktivnih modela za procjenu njihove uspješnosti. Daljnje istraživanje i usavršavanje modela moglo bi doprinijeti boljem razumijevanju čimbenika koji utječu na uspjeh filma te olakšati donošenje odluka u filmskoj industriji.